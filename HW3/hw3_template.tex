\documentclass{article}
\usepackage[letterpaper]{geometry}
\geometry{verbose,tmargin=1in,bmargin=1in,lmargin=1in,rmargin=1in}

\usepackage[utf8]{inputenc}
\usepackage{amsmath}
\usepackage{listings}
\usepackage{graphicx}
\usepackage{enumitem}

\title{CIS 419/519: Homework 1}
\author{\{Your name here\}}
\date{}

\begin{document}
    \maketitle
    Although the solutions are entirely my own, I consulted with the following people and sources while working on this homework: \{Names and sources here\}
    
    \section{Decision Tree Learning}
        \begin{enumerate}[label=\alph*.]
            \item % a
            Show your work:
            \begin{equation*}
                \mathit{InfoGain}(\mathit{PainLocation}) =
            \end{equation*}
            \begin{equation*}
                \mathit{InfoGain}(\mathit{Temperature}) =
            \end{equation*}
            
            \item % b
            Show your work:
            \begin{equation*}
                \mathit{GainRatio}(\mathit{PainLocation}) =
            \end{equation*}
            \begin{equation*}
                \mathit{GainRatio}(\mathit{Temperature}) =
            \end{equation*}
            
            \item % c 
            ~\\
            
            \includegraphics[width=0.8\textwidth]{example-image}
                        
            \item % d
            Yes/No because...
        \end{enumerate}
        
       \section{Decision Trees \& Linear Discriminants [CIS 519 ONLY]}
        
        A decision tree can include oblique splits by...
        
        
        \section{Programming Exercises}
        \textbf{Features}: What features did you choose and how did you preprocess them?
        
        \noindent\textbf{Parameters}: What parameters did you use to train your best decision tree
        
        \noindent\textbf{Performance Table}: 
        \begin{center}
            \begin{tabular}{|c|c|c|}
                \hline
                Feature Set & Accuracy & Conf. Interval [519 ONLY]\\
                \hline
                DT 1 & a & b  \\
                DT 2 & a & b  \\
                DT 3 & a & b  \\
                \hline
        \end{tabular}
                \end{center}
        
        
        
        \textbf{Conclusion}: What can you conclude from your experience?
        
\end{document}